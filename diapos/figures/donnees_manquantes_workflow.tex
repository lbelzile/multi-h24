\documentclass{standalone}
\usepackage{tikz}
\usepackage{amsmath}
\renewcommand{\familydefault}{\sfdefault}
\usetikzlibrary{positioning}
\begin{document}
\tikzset{
    > = stealth,
    every node/.append style = {
        draw = none,
    },
    every path/.append style = {
        arrows = ->,
    },
    hidden/.style = {
        shape = circle,
        draw = black!80,
        fill = white,
        inner sep = 5pt,
        text = white
    }
}
\tikz{
    \node (a) at (-4,0) [hidden]{};
    \node (b) at (-1,2) [hidden] {};
    \node (c) at (0.5,0) [inner sep = 35pt] {$\cdots$};
    \node (d) at (-1,-2) [hidden] {};
    \node (e) at (2, 2) [hidden] {};
%     \node (f) at (1, 0) [hidden] {};
    \node (g) at (2,-2) [hidden] {};
    \node (h) at (5, 0) [hidden] {};
    \node (i) at (-1,1) [hidden] {};
    \node (j) at (-1,-1) [hidden] {};
    \node (k) at (2,1) [hidden] {};
    \node (l) at (2,-1) [hidden] {};
    \path (a) edge (b);
    \path (a) edge (c);
    \path (a) edge (d);
    \path (b) edge (e);
    \path (c) edge (h);
    \path (d) edge (g);
    \path (e) edge (h);
%     \path (f) edge (h);
    \path (a) edge (i);
    \path (a) edge (j);
    \path (i) edge (k);
    \path (j) edge (l);
    \path (k) edge (h);
    \path (l) edge (h);
    \path (g) edge (h);
    \node at (-4,-3) {données\vphantom{p}};
     \node at (-4,-3.4) {incomplètes};
     \node at (-1,-3) {imputation\vphantom{d}};
     \node at (2,-3) {modélisation\vphantom{p}};
     \node at (5,-3)  {combinaison\vphantom{p}};
     \node at (5,-3.4)  {des résultats};
}

\end{document}
